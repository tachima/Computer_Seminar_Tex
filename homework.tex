\documentclass[a4paper,12pt]{article}

\begin{document}

\begin{center}
\Huge{コンピュータゼミ2018宿題}
\end{center}
    
    \section{1章}
    私達の研究室ではおもにシステムやソフトウェア
    の信頼性に関する研究を行っています。
    おもにそれらを確率論によってモデル化し、
    解析することで信頼性の評価を行います。
    
    具体的には以下のような確率過程を
    用いることが多いです。
    
    \begin{itemize}
        \item NHPP
        \item CTMC
    \end{itemize}    
        
    \section{2章}
    卒業論文や現行の作成のさいには
    \LaTeX を使って文章を作成します。
    \LaTeX は数式などを含むような
    文章をきれいに作成するための言語です。
    
    \section{3章}
    確率変数$X$が指数分布に従うとき、
    その分布関数$F_X(t)$と密度関数
    $f_X(t)$は、
    
    \begin{eqnarray}
    F_X(t)&=&1-\mathrm{e}^{-\lambda t} \\
    f_X(t)&=&\lambda \mathrm{e}^{-\lambda t} 
    \end{eqnarray}
    
    となる。またその期待値は定義より、
    
    \begin{eqnarray}
    E[X]&=&\int_0^{\infty} tf_X(t)dt \nonumber \\
        &=&[ (1-\mathrm{e}^{-\lambda t})t]_0^{\infty}-\int_0^\infty(1-\mathrm{e}^{-\lambda t})dt \nonumber \\
        &=&[ (1-\mathrm{e}^{-\lambda t})t]_0^{\infty}-[ t+\frac{1}{\lambda}\mathrm{e}^{-\lambda t}]_0^{\infty} \nonumber \\
        &=&\frac{1}{\lambda}
    \end{eqnarray}
    となる。(extra\,宿題:式(3)を導入してみよう
    \,ヒント:部分積分)
    
    \section{4章}
    表をつくることもできます
    \begin{table}[htb]
    \centering
        \begin{tabular}{|c|c|c|} \hline
            1 & 2 & 3 \\ \hline
            $\alpha$ & $\beta$ & $\gamma$ \\ \hline
        \end{tabular}
    \end{table}

\end{document}
